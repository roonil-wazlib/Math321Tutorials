\documentclass{article}


\usepackage{minted}
\usepackage{fancyhdr}
\usepackage{amsmath,amssymb,amsthm}
\usepackage{graphicx}
\usepackage{pdfpages}
\usepackage[margin=1in]{geometry}
\usepackage{parskip}
\usepackage{caption}
\usepackage{listings}
\usepackage{physics}
\usepackage{mathtools}
\usepackage{xparse}
\usepackage{dsfont}
\usepackage{amsmath}
\usepackage{amssymb}


\title{	
	\normalfont\normalsize
	\vspace{25pt} % Whitespace
	\rule{\linewidth}{0.5pt}\\ % Thin top horizontal rule
	\vspace{20pt} % Whitespace
	{\huge MATH321 Tutorial 2}\\ % The assignment title
	\vspace{12pt} % Whitespace
	\rule{\linewidth}{2pt}\\ % Thick bottom horizontal rule
	\vspace{12pt} % Whitespace
}

\author{\LARGE Emma Hogan \\\\53837798} % Your name


\date{\normalsize\today} % Today's date (\today) or a custom date
\begin{document}
\begin{titlepage}
\maketitle
\thispagestyle{empty}
\end{titlepage}
\section*{Question 1}
\subsection*{Show that char(\(\mathds{Z}_m \oplus \mathds{Z}_n)\) is the least common multiple of \(m\) and \(n\).}
First we show that \(n = char(\mathds{Z}_n)\):
\\\\
\(n\) is the lowest positive integer such that \(n*1 = 0\) in \(\mathds{Z}_n\). Also, \(\forall a \in \mathds{Z}_n\), \(n*a = 0\).
\\\\
Therefore \(char(\mathds{Z}_n) = n\).
\\\\
We want the lowest possible k so that \(\forall a \in \mathds{Z}_m, \b \in \mathds{Z}_n\):
\begin{align*}
char(\mathds{Z}_m \oplus \mathds{Z}_n) &= k\\
k(a,b) = (ka,kb) &= (0,0)\\
\implies ka &= 0\\
\implies kb &= 0\\
\end{align*}
\(k\) must be a multiple of \(m\) and a multiple of \(n\) so that
\begin{align*}
k*1 &= 0 \in \mathds{Z}_m\\
k*1 &= 0 \in \mathds{Z}_n
\end{align*}
So the lowest value k could possible hold is \(lcm(m,n)\).  Let \(k = lcm(m,n)\), with \(k = xn\) and \(k = ym\). We now show this is sufficient.
\begin{align*}
(ka, kb) = (x*n*a,y*m*b) = (x*0, y*0) = (0,0)
\end{align*}
So \(char(\mathds{Z}_m \oplus \mathds{Z}_n) = lcm(m,n)\).
\section*{Question 2}
\subsection*{Determine whether \(3x^2 + 6x - 6\) is irreducible over \(\mathds{Z}\); over \(\mathds{Q}\); over \(\mathds{Z}_5\); over \(\mathds{Z}_{11}\).}
Let \(f(x) = 3x^2 + 6x - 6\).
\\\\
Over \(\mathds{Z}\):
\\\\
3 is not a unit in \(\mathds{Z}\) so \(f(x)\) is reducible because a factor of 3 can be taken out.
\\\\
Over \(\mathds{Q}\):
\\\\
\(\mathds{Q}\) is a field, so since \(f(x)\) has no zeroes in \(\mathds{Q}\), it is irreducible.
\\\\
Over \(\mathds{Z}_5\):
\\\\
\(\mathds{Z}_5\) is a field, so since \(f(x)\) has no zeroes in \(\mathds{Z}_5\), it is irreducible.
\\\\
Over \(\mathds{Z}_{11}\):
\\\\
\(f(x)\) has zeroes \(x=4\) and \(x=5\) in \(\mathds{Z}_{11}\), so \(f(x)\) is reducible over \(\mathds{Z}_{11}\).
\section*{Question 3}
\subsection*{Consider the map \(\varphi:\mathds{Z}[\sqrt(5)] \rightarrow \(M_2(\mathds{Z})\) given by \(\varphi(m + n\sqrt(5) =
\begin{bmatrix}
    m & 5n\\
    n & m\\
\end{bmatrix}\).}
\subsection*{(a) Verify that \(\varphi\) is a ring homomorphism.}
Checking \(\varphi\) is additive:

\begin{align*}
\varphi(m_1 + n_1\sqrt{5}) + \varphi(m_2 + n_2\sqrt{5}) &= 
\begin{bmatrix}
	m_1 & 5n_1\\
	n_1 & m_1\\
\end{bmatrix}
+ \begin{bmatrix}
	m_2 & 5n_2\\
	n_2 & m_2\\
\end{bmatrix}\\
&= \begin{bmatrix}
	m_1 + m_2 & 5n_1 + 5n_2\\
	n_1 + n_2 & m_1 + m_2\\
\end{bmatrix}\\
&= \begin{bmatrix}
	m_1 + m_2 & 5(n_1 + n_2)\\
	n_1 + n_2 & m_1 + m_2\\
\end{bmatrix}\\
&= \varphi(m_1+m_2+(n_1+n_2)\sqrt{5})\\
&= \varphi((m_1 + n_1\sqrt{5}) + (m_2 + n_2\sqrt{5}))
\end{align*}
So \(\varphi\) is additive. Checking \(\varphi\) is multiplicative:

\begin{align*}
\varphi(m_1 + n_1\sqrt{5})*\varphi(m_2+n_2\sqrt{5}) &=
\begin{bmatrix}
	m_1 & 5n_1\\
	n_1 & m_1\\
\end{bmatrix}
* \begin{bmatrix}
	m_2 & 5n_2\\
	n_2 & m_2\\
\end{bmatrix}\\
&= \begin{bmatrix}
	m_1m_2 + 5n_1n_2 & 5m_15n_2 + 5m_2n_1\\
	m_2n_1 + m_1n_2 & 5n_1n_2 + m_1m_2\\
\end{bmatrix}\\
&= \begin{bmatrix}
	m_1m_2 + 5n_1n_2 & 5(m_1n_2 + n_1m_2)\\
	m_1n_2 + n_1m_2 & m_1m_2 + 5n_1n_2\\
\end{bmatrix}\\
&= \varphi((m_1m_2+5n_1n_2) + (m_1n_2 + n_1m_2)\sqrt{5})\\
&= \varphi((m_1 + n_1\sqrt{5}) * (m_2 + n_2\sqrt{5}))
\end{align*}
So \(\varphi\) is multiplicative. Therefore, \(\varphi\) is a homomorphism.
\subsection*{(b) \(S = \left{
\begin{bmatrix}
	a & b\\
	0 & c\\
\end{bmatrix} | a,b,c \in \mathds{Z} \right}\) is a subring of \(M_2(\mathds{Z})\). Find \(\varphi^{-1}(S)\).}
\section*{Question 4}
\subsection{Let R and S be rings. Show that \(R \oplus S\) and \(S \oplus R\) are isomorphic.}
Define \(\varphi : R \oplus S \rightarrow S \oplus R\) by
\begin{align*}
\varphi((a,b)) = (b,a)
\end{align*}
\(\varphi\) is additive because:
\begin{align*}
\varphi((a_1, b_1) + (a_2,b_2)) &= \varphi((a_1 + a_2, b_1 + b_2))\\
&= (b_1 + b_2, a_1 + a_2)\\
&= (b_1, a_1) + (b_2, a_2)\\
&= \varphi(a_1, b_2) + varphi(a_2, b_2)\\
\end{align*}
\(\varphi\) is multiplicative because:
\begin{align*}
\varphi((a_1, b_1)*(a_2, b_2)) &= \varphi((a_1a_2, b_1b_2))\\
&= (b_1b_2, a_1a_2)\\
&= (b_1, a_1) * (b_2, a_2)\\
&= \varphi(a_1, b_1)*\varphi(a_2, b_2)
\end{align*}
Therefore, \(\varphi\) is a homomorphism. Clearly it is also both one-to-one and onto, so \(\varphi\) is an isomorphism. Since there exists an isomorphis between \(R \oplus S\) and \(S \oplus R\), they are isomorphic.
\end{document}
