\documentclass{article}


\usepackage{minted}
\usepackage{fancyhdr}
\usepackage{amsmath,amssymb,amsthm}
\usepackage{graphicx}
\usepackage{pdfpages}
\usepackage[margin=1in]{geometry}
\usepackage{parskip}
\usepackage{caption}
\usepackage{listings}
\usepackage{physics}
\usepackage{mathtools}
\usepackage{xparse}
\usepackage{dsfont}
\usepackage{amsmath}
\usepackage{amssymb}


\title{	
	\normalfont\normalsize
	\vspace{25pt} % Whitespace
	\rule{\linewidth}{0.5pt}\\ % Thin top horizontal rule
	\vspace{20pt} % Whitespace
	{\huge MATH321 Tutorial 1}\\ % The assignment title
	\vspace{12pt} % Whitespace
	\rule{\linewidth}{2pt}\\ % Thick bottom horizontal rule
	\vspace{12pt} % Whitespace
}

\author{\LARGE Emma Hogan \\\\53837798} % Your name


\date{\normalsize\today} % Today's date (\today) or a custom date
\begin{document}
\begin{titlepage}
\maketitle
\thispagestyle{empty}
\end{titlepage}
\section*{Question 1}
\subsection*{Let \(d\) be an integer. Show that \(\mathds{Z}[\sqrt{d}] = \{a + b\sqrt{d}\) \(|\) \(a,b \in \mathds{Z}\}\) is an integral domain.}

\solution
First, observe \(\mathds{Z}[\sqrt{d}]\) is a ring because it is a subset of \(\mathds{C}\) that is non-empty (as \(a, b \in \mathds{Z} \neq \phi\)), and \(\forall a_1, a_2, b_1, b_2 \in \mathds{Z}[\sqrt{d}]\) it is:
\\\\\\
	\hspace*{8mm} 1) Closed under subtraction:
\begin{align*}
  (a_1 + b_1\sqrt{d}) - (a_2 + b_2\sqrt{d}) = (a_1 - a_2) + (b_1 - b_2)\sqrt{d} \in \mathds{Z}[\sqrt{d}]
\end{align*}	

	\hspace*{8mm} 2) Closed under multiplication:
\begin{align*}
	(a_1 + b_1\sqrt{d})(a_2 + b_2\sqrt{d}) = (a_1 a_2 + b_1 b_2 d) + (a_1 b_2 + b_1 a_2)\sqrt{d} \in \mathds{Z}[\sqrt{d}]
\end{align*}


[Note: \(ab,\) \(a-b \in \mathds{Z}\) \(\forall\) \(a, b \in \mathds{Z}\)]
\\\\\\\\
Now, with \(a'=1\), \(b'=0\) we have \(e = 1\) in \(\mathds{Z}[\sqrt{d}]\) such that \(\forall\) \(a+b\sqrt{d}\) \(\in\) \(\mathds{Z}[\sqrt{d}]\):
\begin{align*}
	e(a+b\sqrt{d}) = (a+b\sqrt{d})e = (a+b\sqrt{d})
\end{align*}

So, \(\mathds{Z}[\sqrt{d}]\) has a unity. Also, since we have shown \(\mathds{Z}[\sqrt{d}]\) is a subring of \(\mathds{C}\) and \(\mathds{C}\) does not have any zero divisors, \(\mathds{Z}[\sqrt{d}]\) also can not have any zero divisors.

Therefore, as \(\mathds{Z}[\sqrt{d}]\) is a ring with unity and no zero divisors, it is an integral domain.
\section*{Question 2}
\subsection*{Find all units and zero-divisors in the ring \(\mathds{Z}_4 \oplus \mathds{Z}_3\).}

\solution
Since addition and multiplication in \(\mathds{Z}_4 \oplus \mathds{Z}_3\) are performed pointwise, we need to find units and zero divisors in \(\mathds{Z}_4\) and \(\mathds{Z}_3\) and pair them up.
\\\\
The units of \(\mathds{Z}_4\) and \(\mathds{Z}_3\) are precisely those elements relatively prime to 4 and 3 respectively.

\hspace* \(\mathds{Z}_4\) units = \{1,3\}\\
\hspace* \(\mathds{Z}_3\) units = \{1,2\}

So by considering all possible pairs, we get the units of \(\mathds{Z}_4 \oplus \mathds{Z}_3\) are (1,1), (1,2), (3,1) and (3,2).

Similarly, by constructing Cayley tables we can observe that \(\mathds{Z}_4\) has a zero-divisor of 2, and \(\mathds{Z}_3\) has no zero-divisors. Thus, the only zero-divisor of \(\mathds{Z}_4 \oplus \mathds{Z}_3\) is (2,0).

\pagebreak
\section*{Question 3}
\subsection*{Prove or disprove that}

(a) \(R = \left\{
  \left[ {\begin{array}{cc}
   0 & a \\
   b & a \\
  \end{array} } \right] \rvert\: a,b,c \in \mathds{Q} \right\}
\)

(b) \(S_d = \left\{
  \left[ {\begin{array}{cc}
   a & bd \\
   b & c \\
  \end{array} } \right] \rvert\: a,b \in \mathds{R} \right\}
\)
\subsection*{is a subring of the matrix ring \(M_2(\mathds{R})\), where \(d\) is an arbitrary (but fixed) integer.}
\solution
\\(a) This fails the subring test:

Let \(a_1, a_2, b_1, b_2, c_1, c_2 \in \mathds{Q}\).
\begin{align*}
	\left[ {\begin{array}{cc}
   0 & a_1 \\
   b_1 & c_1 \\
  \end{array} } \right]
  \left[ {\begin{array}{cc}
   0 & a_2 \\
   b_2 & c_2 \\
  \end{array} } \right] = 
  \left[ {\begin{array}{cc}
   a_1 b_2 & a_1 c_1 \\
   c_1 b_2 & b_1 a_2 + c_1 c_2 \\
  \end{array} } \right] \notin R.
\end{align*}
So \(R\) is not a subring of \(M_2(\mathds{R})\) as it is not closed under multiplication.
\\\\
(b) Again we apply the subring test. \(S_d\) is clearly non-empty, so take \(a_1, a_2, b_1, b_2 \in \mathds{R}, d \in \mathds{Z}\).
\begin{align*}
	\left[ {\begin{array}{cc}
   a_1 & b_1 d \\
   b_1 & a_1 \\
  \end{array} } \right] - 
  \left[ {\begin{array}{cc}
   a_2 & b_2 d \\
   b_2 & a_2 \\
  \end{array} } \right] = 
  \left[ {\begin{array}{cc}
   a_1 - a_2 & b_1 d - b_2 d \\
   b_1 - b_2 & a_1 - a_2 \\
  \end{array} } \right] \in S_d
\end{align*}
So \(S_d\) is closed under subtraction.
\begin{align*}
\left[ {\begin{array}{cc}
   a_1 & b_1 d \\
   b_1 & a_1 \\
  \end{array} } \right]
  \left[ {\begin{array}{cc}
   a_2 & b_2 d \\
   b_2 & a_2 \\
  \end{array} } \right] = 
  \left[ {\begin{array}{cc}
   a_1 a_2 + b_1 d b_2 & a_1 b_2 d + b_1 d a_2 \\
   b_1 a_2 + a_1 b_2  & a_1 a_2 + b_1 b_2 d \\
  \end{array} } \right] = \left[ {\begin{array}{cc}
   a_1 a_2 + b_1 b_2 d & (a_1 b_2 + b_1 a_2)d \\
   a_1 b_2 + b_1 a_2 & a_1 a_2 + b_1 b_2 d \\
  \end{array} } \right] \in S_d
\end{align*}
(By taking \(a = a_1 a_2 + b_1 b_2 d, b = a_1 b_2 + b_1 a_2\))
\\\\
So \(S_d\) is closed under multiplication as well, and is therefore a subring of \(M_2(\mathds{R})\).

\section*{Question 4}
\subsection*{Find units and zero divisors in the RING examples in Q3. Do you find them all? Does the answer depend on the value of \(d\)?}
\solution
The units of \(S_d\) are the invertible matrices of the form
\begin{align*}
\left[ {\begin{array}{cc}
   a & bd \\
   b & a \\
  \end{array} } \right]
\end{align*}
These are invertible when
\begin{align*}
a^2 - b^2 d \neq 0\\
a^2 \neq b^2 d\\
a \neq \pm b\sqrt{d}
\end{align*}
Zero-divisors are any elements
\begin{align*}
\left[ {\begin{array}{cc}
   a_1 & b_1 d \\
   b_1 & a_1 \\
  \end{array} } \right]
\end{align*}
Such that for some \(a_2, b_2 \in \mathds{R}\)
\begin{align*}
\left[ {\begin{array}{cc}
   a_1 & b_1 d \\
   b_1 & a_1 \\
  \end{array} } \right] 
  \left[ {\begin{array}{cc}
   a_2 & b_2 d \\
   b_2 & a_2 \\
  \end{array} } \right] = 
  \left[ {\begin{array}{cc}
   0 & 0 \\
   0 & 0 \\
  \end{array} } \right]
\end{align*}
I'm too lazy to solve simultaneous equations or mess around with null spaces, so instead I will take a shortcut by observing that zero-divisors can never be units. So, by the working earlier used to classify units, we know that all zero divisors must have \(a = \pm b\sqrt{d}\). Note this is a sufficient but not necessary condition for the zero divisors, so we are not done. We have however, eliminated some variables.
\\\\
By substituting \(a_1 = \pm b_1 \sqrt{d}\) and \(a_2 = \pm b_2 \sqrt{d}\) into the product calculated in Question 3 and solving each entry for 0, we find that for any matrix of the form
\begin{align*}
\left[ {\begin{array}{cc}
   \pm b_1 \sqrt{d} & b_1 d \\
   b_1 & \pm b_1 \sqrt{d} \\
  \end{array} } \right]
\end{align*}
We can calculate \(a_2 = \mp b_2\sqrt{d}\) and find another matrix of the form
\begin{align*}
\left[ {\begin{array}{cc}
   \mp b_2 \sqrt{d} & b_2 d \\
   b_2 & \mp b_2 \sqrt{d} \\
  \end{array} } \right]
\end{align*}
And their product will be 0 as shown. 
\begin{align*}
\left[ {\begin{array}{cc}
   \pm b_1\sqrt{d} & b_1 d \\
   b_1 & \pm b_1\sqrt{d} \\
  \end{array} } \right]
\left[ {\begin{array}{cc}
   \mp b_2\sqrt{d} & b_2 d \\
   b_2 & \mp b_2\sqrt{d} \\
  \end{array} } \right] = 
  \left[ {\begin{array}{cc}
   -b_1 b_2 d + b_1 b_2 d & \mp b_1 b_2 d \sqrt{d} \pm b_1 b_2 d \sqrt{d} \\
    \mp b_1 b_2 \sqrt{d} \pm b_1 b_2 \sqrt{d} & b_1 b_2 d - b_1 b_2 d \\
  \end{array} } \right] = 
  \left[ {\begin{array}{cc}
   0 & 0 \\
   0 & 0 \\
  \end{array} } \right]
\end{align*}
The condition on \(d\) is that \(S_d\) is only defined over the reals, so \(d\geq 0\). If \(d<0\) then every element in \(S_d\) is invertible, and there are no zero-divisors.
\\\\
Interestingly, despite me saying not being a unit was only a necessary condition for being a zero-divisor, in this ring (ignoring the zero matrix), it turned out to be sufficient. I have not convinced myself of a reason why this is the case - the ring is commutative, but not finite so the result shown in class doesn't apply. I feel like I'm missing something.
\end{center}
\end{document}