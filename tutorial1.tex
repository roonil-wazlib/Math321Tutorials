\documentclass{article}


\usepackage{minted}
\usepackage{fancyhdr}
\usepackage{amsmath,amssymb,amsthm}
\usepackage{graphicx}
\usepackage{pdfpages}
\usepackage[margin=1in]{geometry}
\usepackage{parskip}
\usepackage{caption}
\usepackage{listings}
\usepackage{physics}
\usepackage{mathtools}
\usepackage{xparse}
\usepackage{dsfont}
\usepackage{amsmath}
\usepackage{amssymb}


\title{	
	\normalfont\normalsize
	\vspace{25pt} % Whitespace
	\rule{\linewidth}{0.5pt}\\ % Thin top horizontal rule
	\vspace{20pt} % Whitespace
	{\huge MATH321 Tutorial 1}\\ % The assignment title
	\vspace{12pt} % Whitespace
	\rule{\linewidth}{2pt}\\ % Thick bottom horizontal rule
	\vspace{12pt} % Whitespace
}

\author{\LARGE Emma Hogan \\\\53837798} % Your name


\date{\normalsize\today} % Today's date (\today) or a custom date
\begin{document}
\begin{titlepage}
\maketitle
\thispagestyle{empty}
\end{titlepage}
\section*{Question 1}
\subsection*{Let \(d\) be an integer. Show that \(\mathds{Z}[\sqrt{d}] = \{a + b\sqrt{d}\) \(|\) \(a,b \in \mathds{Z}\}\) is an integral domain.}

\solution
First, observe \(\mathds{Z}[\sqrt{d}]\) is a ring because it is a subset of \(\mathds{R}\) that is non-empty (as \(a, b \in \mathds{Z} \neq \phi\)), and \(\forall a_1, a_2, b_1, b_2 \in \mathds{Z}[\sqrt{d}]\) it is:
\\\\\\
	\hspace*{8mm} 1) Closed under subtraction:
\begin{align*}
  (a_1 + b_1\sqrt{d}) - (a_2 + b_2\sqrt{d}) = (a_1 - a_2) + (b_1 - b_2)\sqrt{d} \in \mathds{Z}[\sqrt{d}]
\end{align*}	

	\hspace*{8mm} 2) Closed under multiplication:
\begin{align*}
	(a_1 + b_1\sqrt{d})(a_2 + b_2\sqrt{d}) = (a_1 a_2 + b_1 b_2 d) + (a_1 b_2 + b_1 a_2)\sqrt{d} \in \mathds{Z}[\sqrt{d}]
\end{align*}


[Note: \(ab,\) \(a-b \in \mathds{Z}\) \(\forall\) \(a, b \in \mathds{Z}\)]
\\\\\\\\
Now, with \(a=1\), \(b=0\) we have \(e = 1\) in \(\mathds{Z}[\sqrt{d}]\) such that \(\forall\) \(a+b\sqrt{d}\) \(\in\) \(\mathds{Z}[\sqrt{d}]\):
\begin{align*}
	e(a+b\sqrt{d}) = (a+b\sqrt{d})e = (a+b\sqrt{d})
\end{align*}

So \(\mathds{Z}[\sqrt{d}]\) has a unity. Also, since we have shown \(\mathds{Z}[\sqrt{d}]\) is a subring of \(\mathds{R}\) and \(\mathds{R}\) does not have any zero divisors, \(\mathds{Z}[\sqrt{d}]\) also can not have any zero divisors.

Therefore, as \(\mathds{Z}[\sqrt{d}]\) is a ring with unity and no zero divisors, it is an integral domain.
\section*{Question 2}
\subsection*{Find all units and zero-divisors in the ring \(\mathds{Z}_4 \oplus \mathds{Z}_4\).}

\solution


\section*{Question 3}
\subsection*{Prove or disprove that}
\solution
\\
\\
\\
Since \(\sum |x_n|\) converges, we know that $\lim_{n\to\infty} |x_n|$\(=0\). So, by definition of the limit, we know \(\exists\) \(N\) such that \(\forall\) \(n\) \(\geq N\), \(|x_n| < 1\).
\\
\\
We can remove a finite number of terms for the start of a series without changing its behaviour. So, we consider \(\sum_{n=N}^{\infty} |x_n|\), so \(|x_n| < 1\) for every term in the sequence.
\\
\\
Note that since \(x_n^2 \geq 0\), \(|x_n|^2 = x_n^2 = |x_n^2|\). 
\\
\\
\(|x_n| < 1\) so \(|x_n|^2 = |x_n^2| < 1\).
\\
\\
\(|x_n^2| < |x_n| \leq |x_n|\) \(\forall n \geq N\), so, since \(\sum_{n=N}^{\infty} |x_n|\) is convergent, \(\sum_{n=N}^{\infty} |x_n^2|\) is also convergent. Once again, N is just a finite number so removing the first N terms has not changed the behaviour of the series.
\\
\\
Therefore \(\sum x_n^2\) is absolutely convergent, which implies it is convergent.
\\
\\
\\
\section*{Question 4}
\subsection*{Prove, from the definition of limit, that \lim_{n\to\infty} x^2+x = 2.}
\solution
\\\\
The limit definition states:
\\\\
\(lim_{x\to p} f(x) = L\) \(\Rightarrow\) \(\forall\) \(\mathcal{E} > 0\), \(\exists\) \(\delta > 0\) such that, \(\forall x \in \mathds{R}\), \(0 < |x-p| < \delta\) we have \(x\) in the domain of \(f\) and \(|f(x)-L| < \mathcal{E}\).
\\
\\
So, to show \(lim_{x\to p} (x^2 + x - 2)\), we need to find \(\delta > 0\) such that, \(\forall\) \(x \in \mathds{R}\), \(0<|x-1|<\delta\), we have \(|x^2+x-2| < \mathcal{E}\) \(\forall\) \(\mathcal{E}\).
\\
\\
Choose \(\delta = min\left\{\frac{\mathcal{E}}{4},1\right\}\).
\\
\begin{align*}
  |x^2 + x - 2| &= |(x-1)(x-2)| \\
  &= |x-1||x-2|\\
\end{align*}
From our definition, we already have \(|x-1| < \delta < \frac{\mathcal{E}}{4}\). So,
\\
\begin{align*}
  |x-2| &= |x-1+3| \\
  &= |(x-1)+3| \\
  &\leq |x-1| + |3| \\
\end{align*}
By the triangle inequality. Now using \(|x-1| < \delta\), with \(\delta = min\left\{\frac{\mathcal{E}}{4},1\right\}\), we get

\begin{align*}
	|x-2| &\leq |x-1| + 3 \\
	&< 1 + 3 \\
	&< 4 \\
\end{align*}
Now we have
\begin{align*}
	|x^2 + x - 2| &= |(x-1)||(x-2)| \\
	&< \frac{\mathcal{E}}{4}\cdot4 \\
	&= \mathcal{E}\\
\end{align*}
So we have found a \(\delta\) that satisfies the definition of the limit \(\forall\) \(\mathcal{E}\).
\section*{Question 5}
\subsection*{Let $f$ be a continuous function on $(-1,1)$ with $f(0) = 1$. Show that there exists $\mathcal{E} >$ 0 such that for all $x$ with $|x| < \mathcal{E}$ we have $f(x) > 2/3$. }
\solution
\\
Since \(f(x)\) is continuous at 0, we know that:
\\
\begin{align*}
	\lim_{x\to 0} f(x) = 1 \\
\end{align*}
From the definition of the limit, we have some \(\delta > 0\) such that \(\forall\) \(x \in \mathds{R}\), \(0<|x|<\delta\), we have \(x \in D(f)\) and \(|f(x) - 1| < \frac{1}{3}\).
\\
\begin{align*}
	|f(x) - 1| &< \frac{1}{3} \\
	\Rightarrow \frac{2}{3} < f(x) &< \frac{4}{3}
\end{align*}
So, by setting \(\mathcal{E} = \delta\), we have \(\mathcal{E} > 0\) such that \(f(x) > \frac{2}{3}\) \(\forall\) \(x\), \(0<|x|<\mathcal{E}\).     (\(\star\))
\\
\\
Since \(f\) is continuous with \(f(0)=1\), \(|x|=0<\mathcal{E}\) works (since \(f(|0|)=1>\frac{2}{3}\)), so we can state that (\(\star\)) holds \(\forall\) \(x\) with \(|x| < \mathcal{E}\) (including \(|x|=0\)).
\\
\\
\end{center}
\end{document}
